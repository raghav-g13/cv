%%%%%%%%%%%%%%%%%%%%%%%%%%%%%%%%%%%%%%%
% This is a modified ONE COLUMN version of
% the following template:
% 
% Deedy - One Page Two Column Resume
% LaTeX Template
% Version 1.1 (30/4/2014)
%
% Original author:
% Debarghya Das (http://debarghyadas.com)
%
% Original repository:
% https://github.com/deedydas/Deedy-Resume
%
% IMPORTANT: THIS TEMPLATE NEEDS TO BE COMPILED WITH XeLaTeX
%
% This template uses several fonts not included with Windows/Linux by
% default. If you get compilation errors saying a font is missing, find the line
% on which the font is used and either change it to a font included with your
% operating system or comment the line out to use the default font.
% 
%%%%%%%%%%%%%%%%%%%%%%%%%%%%%%%%%%%%%%
% 
% TODO:
% 1. Integrate biber/bibtex for article citation under publications.
% 2. Figure out a smoother way for the document to flow onto the next page.
% 3. Add styling information for a "Projects/Hacks" section.
% 4. Add location/address information
% 5. Merge OpenFont and MacFonts as a single sty with options.
% 
%%%%%%%%%%%%%%%%%%%%%%%%%%%%%%%%%%%%%%
%
% CHANGELOG:
% v1.1:
% 1. Fixed several compilation bugs with \renewcommand
% 2. Got Open-source fonts (Windows/Linux support)
% 3. Added Last Updated
% 4. Move Title styling into .sty
% 5. Commented .sty file.
%
%%%%%%%%%%%%%%%%%%%%%%%%%%%%%%%%%%%%%%%
%
% Known Issues:
% 1. Overflows onto second page if any column's contents are more than the
% vertical limit
% 2. Hacky space on the first bullet point on the second column.
%
%%%%%%%%%%%%%%%%%%%%%%%%%%%%%%%%%%%%%%

\documentclass[]{deedy-resume-openfont}


\begin{document}

%%%%%%%%%%%%%%%%%%%%%%%%%%%%%%%%%%%%%%
%
%     LAST UPDATED DATE
%
%%%%%%%%%%%%%%%%%%%%%%%%%%%%%%%%%%%%%%
%\lastupdated

%%%%%%%%%%%%%%%%%%%%%%%%%%%%%%%%%%%%%%
%
%     TITLE NAME
%
%%%%%%%%%%%%%%%%%%%%%%%%%%%%%%%%%%%%%%


\namesection{Raghav}{Gupta}{ \urlstyle{same}\href{https://raghav-g13.github.io}{Website} | 
\href{https://github.com/raghav-g13}{GitHub} | \href{https://www.linkedin.com/in/raghavgupta1303/}{LinkedIn}\\
\href{mailto:raghavgupta@berkeley.edu}{raghavgupta@berkeley.edu} | 424.272.7363
}

\section{Education}


\runsubsection{University of California, Berkeley}
\descript{| PhD in Computer Science}
\location{Aug 2023 - Present}
Advisor: Prof. Borivoje Nikolic
\vspace*{3pt}
\\
\descript{Coursework: } Architectures and Systems for Warehouse-Scale Computers (A)

\vspace{6pt}

\runsubsection{University of California, Berkeley}
\descript{| BS in Electrical Engineering and Computer Science}
\location{Aug 2019 - May 2023 | GPA: 3.95/4.0}
High Honors | Dean's List
\vspace*{3pt}
\\
\descript{Coursework: } Computer Architecture and Engineering (A)
\textbullet{}
Operating Systems and System Programming (A)
\textbullet{}
Introduction to Digital Design and Integrated Circuits (A+)
\textbullet{}
22nm SoC for IoT (A)
\textbullet{}
Advanced Topics in Computer Systems (A-)
\textbullet{}
Programming Languages and Compilers (A)
% \textbullet{}
% Introduction to Robotics
% \textbullet{}
% Quantum Mechanics
% \textbullet{}
% Efficient Algorithms and Intractable Problems
% \textbullet{}
% Signals and Systems
% \textbullet{}
% Designing Information Devices and Systems I, II
% \textbullet{}
% Discrete Mathematics and Probability Theory
% \textbullet{}
% Teaching Techniques for Computer Science
% \textbullet{}
% Social Implications of Computer Technology

\sectionsep

\section{Research Interests: Computer Architecture}
Hyperscale Architecture and Systems 
\textbullet{}
Hardware Design Methodology
% \textbullet{} 
% Hardware/Software Co-design 
% \textbullet{}
% Profile-Guided Methods
\sectionsep

\section{Research Experience}

\runsubsection{SLICE Lab at UC Berkeley}
\descript{| Student Researcher}
\location{Jan 2021 – Present | Berkeley, CA |  \href{https://slice.eecs.berkeley.edu}{\bf Website}}

% \vspace{\topsep} % Hacky fix for awkward extra vertical space
% Working on computer architecture topics with graduate students in Prof. Krste Asanovic's and Prof. Borivoje Nikolic's groups. 

\vspace{\topsep} % Hacky fix for awkward extra vertical space

\descript{Hybrid Simulation/Emulation | {\bf Collaboration with AMD/Xilinx} | {\bf Jun 2024 - Present}}
{\normalsize
\begin{tightemize}
\item Building a hybrid simulation/emulation platform atop FireSim to enable design verification and early DSE 
\item Initial prototype allows system bus modules to be simulated in C++ while the rest of the SoC is emulated on an FPGA
\item Initial results show O(10 MHz) simulation frequency when running Linux+Coremark and snooping on SoC memory requests in software 
\end{tightemize}
}

\descript{Multi-level Simulation | {\bf Oct 2023 - Apr 2024}}
{\normalsize
\begin{tightemize}
\item Helped build a multi-level simulator trading off speed and fidelity in a hierarchy of simulators for DSE on large workloads
\item Used program analysis to identify critical portions of an execution trace to execute at high fidelity
\item Investigated warm-up strategies to approximate cache state 
\end{tightemize}
}


% \descript{Accelerating Hyperscale RPCs | {\bf In-progress}}
% {\normalsize
% \begin{tightemize}
% \item Investigating orchestration to offload RPCs to a collection of accelerators
% \item Working on porting the RPC software stack to RISC-V
% \end{tightemize}
% }

\descript{Extending FirePerf to Userspace |  \href{https://dl.acm.org/doi/pdf/10.1145/3373376.3378455}{\bf Original Paper} | \href{https://docs.google.com/gview?url=https://github.com/Quizenger/cv/raw/main/resources/FirePerf_Poster.pdf}{\bf Poster} | \href{https://docs.google.com/gview?url=https://github.com/Quizenger/cv/raw/main/resources/FirePerf_Paper.pdf}{\bf Userspace Paper}}

\begin{tightemize}
{\normalsize
\item FirePerf is an FPGA-accelerated full-system hardware/software performance profiling utility built atop FireSim
\item Provides high-fidelity out-of-band introspection with call stack reconstruction and Flame Graph generation
\item Extended FirePerf to support userspace profiling by instrumenting Rocket Chip trace port, obtaining DWARF + hex information, constructing a user binary search space and matching traces to it
}
\end{tightemize}

\descript{FireMarshal Networking |  \href{https://ieeexplore.ieee.org/abstract/document/9408192/}{\bf Original Paper}}
\begin{tightemize}
{\normalsize
\item FireMarshal is an open-source software workload management tool that automates bootable workload image generation and evaluation on FireSim, QEMU, and Spike platforms
\item Implemented running multi-node workloads concurrently in QEMU
\item Implemented a network interface using Virtual Distributed Ethernet for multi-node workloads emulated in QEMU
}
\end{tightemize}

% \vspace{\topsep}

\sectionsep

%\topsep
% \vspace{\topsep}
\section{Work Experience}

\vspace{\topsep}
\runsubsection{AMD/Xilinx}
\descript{| Research and Advanced Development Intern}
\location{Jun 2024 – Dec 2024 | San Jose, CA }
\vspace{\topsep} % Hacky fix for awkward extra vertical space
\descript{Hybrid Simulation/Emulation}
{\normalsize
\begin{tightemize}
\item Implemented and evaluated two strategies to build a fast and flexible hybrid simulation/emulation platform
\item Ported an internal co-simulation framework and an open-source emulation framework to an internal emulation machine
\end{tightemize}
}

\vspace{\topsep}
\runsubsection{Omnistrate}
\descript{| Software Engineering Intern}
\location{May 2023 – July 2023 | Remote }
\vspace{\topsep} % Hacky fix for awkward extra vertical space
\descript{Monitoring for Stateful Containerized Cloud Applications}
{\normalsize
\begin{tightemize}
\item Developed monitoring infrastructure for stateful containers using the sidecar pattern
\item Implemented support for application-specific probes and process metadata checks to trigger alerts and fail-over
\item Improved reliability at scale for a public database provider's DBaaS offering
\end{tightemize}
}

\vspace{\topsep}
\runsubsection{NVIDIA}
\descript{| Power Architect Intern}
\location{May 2022 – Aug 2022 | Santa Clara, CA }
\vspace{\topsep} % Hacky fix for awkward extra vertical space
\descript{CPU Power Modeling for Gaming Workloads}
{\normalsize
\begin{tightemize}
\item Developed a strategy for application-specific CPU power modeling on production silicon
\item Targeted characteristics of gaming workloads and isolated static and dynamic power costs
\item Automated data collection, processing, visualization, modeling and summary statistics
\end{tightemize}
}
\sectionsep

% \vspace*{3pt}
% \\
% \descript{Research Interests: Computer Architecture}
% \textbullet{}
% Cache and Memory Structures

\section{Teaching Experience}

\runsubsection{CS 152/252A -  Computer Architecture and Engineering }
\descript{| Teaching Assistant}
\location{Spring 2024 |  \href{https://inst.eecs.berkeley.edu/~cs152/sp24/}{\bf Website}}
\vspace{\topsep} % Hacky fix for awkward extra vertical space
\begin{tightemize}
\item Led overall organization of a course catered towards 130+ engineering graduates and upper-division undergraduates
\item Meticulously designed and debugged problems for exams and homework assignments
\item Helped students learn about pipelining, OoO core design, the memory hierarchy, virtual memory, branch prediction, etc., in weekly office hours and by answering on the course Q/A forum
\end{tightemize}
\sectionsep

\runsubsection{EE290C - 22nm SoC for IoT}
\descript{| Teaching Assistant}
\location{Fall 2022 - Spring 2023}
\vspace{\topsep}
\begin{tightemize}
\item Developed teaching resources for RISC-V SoC tapeout, on topics such as the use of RTL generators (especially custom accelerators) and verification in simulation, that were shared as part of Intel's University Shuttle Program
\item Managed and supported 3 teams across 2 chips as they designed specialized modules for sparse-dense matrix multiplication, near-cache compute, and Extended Kalman Filter
\end{tightemize}
\sectionsep

\runsubsection{EECS 151/251A -  Introduction to Digital Design
and Integrated Circuits }
\descript{| Teaching Assistant}
\location{Fall 2022 |  \href{https://inst.eecs.berkeley.edu/~eecs151/fa22/}{\bf Website}}
\vspace{\topsep} % Hacky fix for awkward extra vertical space
\begin{tightemize}
\item Taught 1 weekly office hour and 1 weekly lab section using Xilinx PYNQ FPGAs for Berkeley's upper division VLSI course catered towards undergraduates and graduates
\item Helped students learn hardware design methods and principles using Verilog, VCS, and waveforms as they implemented FSMs, audio synthesis circuits, UART modules and FIFOs on FPGAs
\item Supported students in applying architectural and microarchitectural concepts to design a 3+ stage pipelined RISC-V datapath with L1 data cache, UART/MMIO and optimize it for timing
\item Designed a memory controller lab with specification and testbenches as an introduction to working with synchronous memories
\end{tightemize}
\sectionsep

\runsubsection{EECS16A - Designing Information Devices and Systems I}
\descript{|  Lab Teaching Assistant and Head Lab Content/Development}
\location{Summer 2020 – Spring 2022 | \href{https://inst.eecs.berkeley.edu/~ee16a/sp22/}{\bf Website}}
\vspace{\topsep} % Hacky fix for awkward extra vertical space
\begin{tightemize}
\item Taught a weekly lab section of 50+ engineering undergraduates for Berkeley's introductory EECS course for 5 semesters
\item Assisted students in building a single pixel camera, resistive and capacitive touchscreens, and an acoustic positioning system
\item Trained lab staff and managed lab content for 1000+ students, revamped existing labs, developed new labs and managed the shift to online labs while achieving the requisite learning goals. 
\end{tightemize}
\sectionsep

\runsubsection{CS61C - Great Ideas in Computer Architecture}
\descript{| Teaching Assistant}
\location{Summer 2021 | \href{https://cs61c.org/su21}{\bf Website}}
\vspace{\topsep} % Hacky fix for awkward extra vertical space
\begin{tightemize}
\item Taught 2 discussion sections and 1 office hour weekly for Berkeley's introductory computer architecture and systems course
\item Developed content and infrastructure for projects on C programming and Logisim datapath design
\item Helped design labs on Logisim datapath design, pipelining, and caches
\end{tightemize}
\sectionsep

\pagebreak

\section{Projects}

\runsubsection{Bearly ML: SoC for Machine Learning}
\descript{| EE290C}
\location{ Jan 2022 – May 2022 | \href{https://hc2023.hotchips.org/assets/program/posters/HC2023.UCBerkeley.YufengChi.Poster.v06.pdf}{\bf Poster}}
\begin{tightemize}
\item Designed, verified, and taped-out a RISC-V SoC with ML/DSP accelerators in Intel 16 technology using the Chipyard environment 
\item Member of 5-person team that developed a sparse-dense matrix multiplication accelerator using the Rocket Custom Coprocessor (RoCC) interface
\item Tightly-coupled with a 5 stage, in-order Rocket core with 16KB L1 DCache, 4KB L1 ICache
\item Wrote Chisel RTL, performed verification in functional simulation with unit-testing and integration testing with bare-metal C tests, and resolved floor-planning Design Rule Violations
\item Maximum frequency: 500 MHz, Achievable Throughput: 2.6 uint8 GOPS
\end{tightemize}
\sectionsep


% \runsubsection{RISC-V Core on FPGA}
% \descript{| EECS151}
% \location{ Oct 2021 – Dec 2021  | \href{https://docs.google.com/gview?url=https://github.com/Quizenger/cv/raw/main/resources/EECS_251A_Lab_Report.pdf}{\bf Report}}
% \begin{tightemize}
% \item Team of 2 that designed a 3-stage RISC-V core with UART/MMIO, polyphonic audio synthesizer and BHT-based branch predictor in Verilog
% \item Won class-wide Apple design contest for performance at a frequency of 70 MHz with a CPI of 1.16 on the matrix multiplication kernel
% \end{tightemize}
% \sectionsep

% \runsubsection{Branch Predictor}
% \descript{| Lab Project for CS152}
% \location{ Mar 2021 – Apr 2021 | \href{https://docs.google.com/gview?url=https://github.com/Quizenger/cv/raw/main/resources/Lab_3_Report_Open_Ended_Portion.pdf}{\bf Report}}
% \begin{tightemize}
% \item Designed and analyzed an 8192-entry tournament predictor using a Gshare predictor, a bimodal predictor, and an arbiter in a 3-person team
% \item Implemented using the C++ branch predictor framework in BOOM out-of-order core
% \item Performed >25\% better than baseline bimodal predictor
% \end{tightemize}
% \sectionsep

\runsubsection{PintOS}
\descript{| Projects for CS162}
\location{ Jan 2022 – May 2022 }
\begin{tightemize}
\item As a 3-person team, implemented the following features in an x86-based educational OS running as a VM in QEMU/Bochs with extensive debugging in GDB
\item Supported user program execution with syscall argument passing, process control syscalls (halt, exec, wait, exit), basic file operation syscalls, and a floating point unit
\item Implemented multithreading with an efficient sleep timer, strict priority scheduler with priority donation, and a simplified version of the user-level pthread library and user-level synchronization
\item Implemented a Unix FFS-like filesystem with buffer cache, extensible files and support for subdirectories and relative paths
\end{tightemize}
\sectionsep

% \runsubsection{CS61CPU}
% \descript{| Project for CS61C}
% \location{ Nov 2020 }
% \begin{tightemize}
% \item Designed a 2-stage pipelined RISC-V Datapath using Logisim in a team of 2
% \item Optimized control logic by simplifying boolean expressions for each signal
% \item Won class-wide Apple design contest for minimal transistor use
% \end{tightemize}
% \sectionsep

\section{Publications and Presentations}
\vspace{8pt}
\begin{tightemize}
\item Whangbo, Joonho, Edwin Lim, Chengyi Lux Zhang, Kevin Anderson, Abraham Gonzalez,\custombold{ Raghav Gupta,} Nivedha Krishnakumar, et al. \custombold{"FireAxe: Partitioned FPGA-Accelerated Simulation of Large-Scale RTL Designs,"} in Proceedings of the 2024 ACM/IEEE 51st Annual International Symposium on Computer Architecture (ISCA), 2024.
% , pp. 501–515, doi: 10.1109/ISCA59077.2024.00044."
%\item <Anonymized>, <Anonymized>, <Anonymized>, <Anonymized>, <Anonymized>,\custombold{ Raghav Gupta,} <Anonymized>, et al. \custombold{"Partitioning Large-Scale Monolithic Hardware Designs for FireSim."} (Under submission)
% \item Whangbo, J., Lim, E., Zhang, C.L., Anderson, K., Gonzalez, A.,\custombold{ Gupta, R.}, Krishnakumar, N., Karandikar, S., Shao, Y. S., Nikolic, B., \& Asanovic, K. \custombold{Partitioning Large-Scale Monolithic Designs for FireSim.} (Under submission)
\item Chi, Yufeng, Franklin Huang,\custombold{ Raghav Gupta,} Ella Schwarz, Jennifer Zhou, Reza Sajadiany, Animesh Agrawal, et al. \href{https://hc2023.hotchips.org/assets/program/posters/HC2023.UCBerkeley.YufengChi.Poster.v06.pdf} {\custombold{"A Heterogeneous RISC-V SoC for ML Applications in Intel 16 Technology."}} Poster presented at HotChips 2023, Stanford, CA, August 27-28, 2023.
\end{tightemize}
\sectionsep

\section{Awards}
\vspace{8pt}
\begin{tightemize}
\item EECS Oustanding TA Award (2020 - 2021)
\item CS61C Project - RISCV CPU in Logisim - Apple Design Award (Fall 2020)
\item EECS151 Project - RISCV Core on FPGA - Apple Design Award (Fall 2021)
\item Edward Frank Kraft Award (Fall 2019)
\item Times of India - Spark Scholarship (2018)
\end{tightemize}
\sectionsep

\section{Organizations}
\runsubsection{Eta Kappa Nu}
\descript{| EECS HONOR
SOCIETY}
\location{Jan 2021 - May 2023}
\vspace{3pt}
\descript{Tutoring Officer}
{\normalsize
Responsible for managing activities of the tutoring committee, such as organizing and running exam review sessions, planning daily office hours, and supporting learning resources, in Spring 2022.
}
\\
\vspace{3pt}
\descript{DeCal Assistant Officer}
{\normalsize
Responsible for facilitating ”Going Down
the EECS Stack”, a course that explores
the different fields within the EECS major, in Fall 2021.
}
\sectionsep


\runsubsection{Curiosity}
\descript{ | Tutor and Co-founder}
\location{May 2020 - Aug 2020}
\vspace{3pt}
\begin{tightemize}
\item Tutored high school students on STEM topics and SAT prep during Summer 2020
\item Raised ~\rupee{15,000} and donated all funds to COVID-19 and flood relief
\end{tightemize}
\sectionsep

\section{Skills}
Python \textbullet{} C/C++ \textbullet{} Git \textbullet{} Linux/Bash \textbullet{} Verilog \textbullet{} Chisel \textbullet{} Go \textbullet{} RISC-V \textbullet{} CUDA \textbullet{} Java \textbullet{} Vivado \textbullet{} Docker \textbullet{} Kubernetes \textbullet{} ROS % \textbullet{} Machine Learning \textbullet{} Deep Learning \textbullet{} Computer Vision
\sectionsep

% \section{References}

% \href{https://www2.eecs.berkeley.edu/Faculty/Homepages/nikolic.html}{\textbf{Prof. Borivoje Nikolic}}

% \href{https://www2.eecs.berkeley.edu/Faculty/Homepages/asanovic.html}{\textbf{Prof. Krste Asanovic}}

% \href{https://www2.eecs.berkeley.edu/Faculty/Homepages/ysshao.html}{\textbf{Prof. Sophia Shao}}

% \section{Languages}
% \begin{minipage}[t]{.6\textwidth}
% \subsection{Programming}
% \location{Over 5000 lines:}
% Java \textbullet{}   Shell \textbullet{} JavaScript \textbullet{} Matlab \textbullet{}
% OCaml \textbullet{} Python \textbullet{} Rails \textbullet{} \LaTeX\ \\ 
% \location{Over 1000 lines:}
% C \textbullet{} C++ \textbullet{} CSS \textbullet{} PHP \textbullet{} Assembly \\
% \location{Familiar:}
% AS3 \textbullet{} iOS \textbullet{} Android \textbullet{} MySQL
% \sectionsep
% \end{minipage}
% \hfill
% \begin{minipage}[t]{.35\textwidth}
% \subsection{Spoken \& Written}
% \location{Native fluency:} English, Spanish\\
% \location{Reading fluency:} Chinese, Japanese\\
% \end{minipage}

\end{document}  \documentclass[]{article}